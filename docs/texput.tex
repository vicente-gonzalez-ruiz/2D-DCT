% Emacs, this is -*-latex-*-

\title{Block-Based DCT (Discrete Cosine Transform)}

\maketitle

\tableofcontents

\section{1D-DCT}

The \href{https://en.wikipedia.org/wiki/Discrete_cosine_transform}{DCT
  (Discrete Cosine Transform)} is an
\href{https://en.wikipedia.org/wiki/Orthonormality}{orthonormal}
\href{https://vicente-gonzalez-ruiz.github.io/transform_coding/}{transform}~\cite{vruiz__transform_coding}.

In the 1D case, the forward DCT for a digital signal $\mathbf{g}_n$ of
length $N$ is defined as~\cite{burger2016digital}
\begin{equation}
  {\mathbf h}_u = \sqrt{\frac{2}{N}}\sum_{n=0}^{N-1}{\mathbf
    g}_n{\mathbf c}_u\cos\Big(\pi\frac{u(2x+1)}{2N}\Big),
\end{equation}
for $0<u<N$, and the inverse transform is
\begin{equation}
  {\mathbf g}_n = \sqrt{\frac{2}{N}}\sum_{u=0}^{N-1}{\mathbf
    h}_u{\mathbf c}_u\cos\Big(\pi\frac{u(2x+1)}{2N}\Big),
\end{equation}
for $0<n<N$, with
\begin{equation}
  {\mathbf c}_u = \left\{
  \begin{array}{ll}
    \frac{1}{\sqrt{2}} & \quad \text{for}~u=0, \\
    1 & \quad \text{otherwise}.
  \end{array}
  \right.
\end{equation}

The transformed signal is a sequence of coefficients ${\mathbf h}_u$
with the same length than the original signal ${\mathbf g}_n$, and the
position of the coefficients in the transform domain indicate the
contribution of such coefficient to the corresponding (in this case,
spatial) frequency. For example, the coefficient at the position 0
(that is commonly refered as DC (Direct Current)) is equal to the
average of the signal. The rest of high-frequency coefficients are
called AC (Alternating Current) coefficients.

The DCT can be also expressed in matrix~\cite{sayood2017introduction}
form as
\begin{equation}
  {\mathbf h} = {\mathbf K}{\mathbf g},
\end{equation}
where ${\mathbf K}$ is the forward transform matrix. The rows of the
transform matrix are often referred to as the basis vectors for the
transform because they form an orthonormal basis set (see these
\href{https://cseweb.ucsd.edu/classes/fa17/cse166-a/lec13.pdf}{slides}),
and the elements of the transformed sequence are often called the
transform coefficients. Obviously, the inverse transform is computed
by
\begin{equation}
  {\mathbf g} = {\mathbf K}^{-1}{\mathbf h},
\end{equation}
where ${\mathbf K}^{-1}$ denotes the inverse matrix of ${\mathbf
  K}$. In the case of the DCT, ${\mathbf K}$ is
\href{https://en.wikipedia.org/wiki/Orthogonal_matrix}{orthogonal} and
therefore, ${\mathbf K}^{-1}={\mathbf K}^{\text T}$, where
$\cdot^{\text T}$ denotes the transpose of ${\mathbf K}$.

\section{3-Components 1D-DCT}

The three components DCT (3C-DCT) is a transform applied to a vector of $3$ elements (in
our case, a $\text{RGB}$ pixel) defined by
\begin{equation}
  \begin{bmatrix}
    \text{DCT}^0 \\
    \text{DCT}^1 \\
    \text{DCT}^2
  \end{bmatrix}
  =
  \begin{bmatrix}
    0.57735027 & 0.70710678 & 0.40824829 \\
    0.57735027 & 0.0 & -0.81649658 \\
    0.57735027 & -0.70710678 & 0.40824829
  \end{bmatrix}
  \begin{bmatrix}
    \text{R} \\
    \text{G} \\
    \text{B}
  \end{bmatrix},
\end{equation}
and the inverse transform by
\begin{equation}
  \begin{bmatrix}
    \text{R} \\
    \text{G} \\
    \text{B}
  \end{bmatrix}
  =
  \begin{bmatrix}
    0.57735027 & 0.57735027 & 0.57735027 \\
    0.70710678 & 0.0 & -0.70710678 \\
    0.40824829 & -0.81649658 & 0.40824829
  \end{bmatrix}
  \begin{bmatrix}
    \text{DCT}^0 \\
    \text{DCT}^1 \\
    \text{DCT}^2
  \end{bmatrix}.
\end{equation}
See the notebook
\href{https://github.com/vicente-gonzalez-ruiz/color_transforms/blob/main/docs/3DCT/notebooks/3DCT_matrix.ipynb}{3-Channels
  DCT} to see how to compute the filter's coefficients.

When applied to the $\text{RGB}$ color domain, we will move from this
domain to the 1D-DCT domain that have also 3 components, that we
will denote by $\text{DCT}^0$, $\text{DCT}^1$ and $\text{DCT}^2$. Notice that if
the decorrelation is effective, most of the energy will be
concentrated in $\text{DCT}^0$, which represents the average energy of
the image (luminance) (see the notebook
\href{https://github.com/vicente-gonzalez-ruiz/color_transforms/blob/main/docs/3DCT/notebooks/3DCT_over_RGB.ipynb}{Removing
  RGB redundancy with the DCT}).

Notice also that the DCT is orthonormal, and therefore, the matrix of
the forward transform is the transpose of the matrix of the backward
transform~\cite{sayood2017introduction}. This also means that the
contribution of the synthesis filters (which define the inverse
transform) to the reconstructed signal are independent and have
exactly the unity gain.\footnote{To find the gains of any 1D transform
we can compute the energy of the signal generated by the inverse
transform of the impulse discrete 1D signal
\begin{equation}
  \delta_{i}(x) = 
  \left\{
  \begin{array}{ll}
    1 & \text{if $i=x$}\\
    0 & \text{otherwise},
  \end{array}
  \right.
\end{equation}
where the
\href{https://en.wikipedia.org/wiki/Energy_(signal_processing)}{energy
  of a discrete signal} ${\mathbf s}$ is defined as
\begin{equation}
  \langle {\mathbf s}, {\mathbf s} \rangle =  \sum_{i}{{\mathbf s}_i^2}.
\end{equation}
}

\section{2D-DCT}

The 2D-DCT is separable, which means that it can be computed by
appliying the 1D-DCT to the two dimensions of the signal (a digital
image, for example). For the inverse 2D-DCT, the procedure is similar,
but appliying the inverse 1D-DWT in reverse order. The
Fig.~\ref{fig:2D-DCT_basis} shows the first 64 2D-DCT basis.

\begin{figure}
  \centering \png{graphics/2D-DCT_basis}{600}
  \caption{First 64 2D-DCT basis functions (see the notebook
    \href{https://github.com/vicente-gonzalez-ruiz/DCT/blob/master/docs/HTML/graphics/2D-DCT_basis.ipynb}{2D-DCT
      Basis}).}
  \label{fig:2D-DCT_basis}
\end{figure}

\section{Scalar quantization in the DCT domain}

Since the DCT is
\href{https://en.wikipedia.org/wiki/Orthonormality}{orthonormal}
(orthogonal\footnote{The contribution of the synthesis filters to the
reconstructed signal are independent.} + unitary\footnote{Unitary
transforms are energy preserving; that is, the sum of the squares of
the transformed sequence is the same as the sum of the squares of the
original sequence.}), the gain of each subband is equal to 1, and
therefore, the quantization error is \emph{amplified} with the same
gain in all the coefficients.

For example, the notebook
\href{https://github.com/vicente-gonzalez-ruiz/color_transforms/blob/main/docs/3DCT/notebooks/3DCT_over_RGB.ipynb}{Removing
  RGB redundancy with the DCT} explores the performence of the
quantization steps pattern
\begin{equation}
  \Delta_{\text{DCT}^0} = \Delta_{\text{DCT}^1} = \Delta_{\text{DCT}^2}.
\end{equation}

However, this does not mean that all the subbands
should be quantized using the same quantization step size $\Delta$,
because some subbands can be compressed more efficienly than others
and their distortion (after quantization) can be different. The
notebook
\href{https://github.com/vicente-gonzalez-ruiz/DCT/blob/master/docs/notebooks/YCoCg_2D_DCT_SQ.ipynb}{Image
  Compression with YCoCg + 2D-DCT} shows how to use RD optimization to
determine, given a target bit-rate, the best combination of
quantization steps per subbands.

\section{RDO (Rate/Distortion Optimization) using scalar quantization}

If we consider that the RD curve can be affeced by the compresibility
of the subbands, the use of scalar quantization open the posibility of using a
different quantization step size for each subband, depending on the
slopes of the corresponding RD points.

The DCT is orthogonal, and this means that we can optimize directly in
the transform domain, using the RD curves of each subband,
independently. Therefore, a better quantization steps pattern can be
found if we select those steps that produce the same slope in each
subband, for a given bit-rate.

\section{References}

\renewcommand{\addcontentsline}[3]{}% Remove functionality of \addcontentsline
\bibliography{maths,data_compression,signal_processing,DWT,image_compression,image_processing}
