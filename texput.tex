% Emacs, this is -*-latex-*-

\title{DCT (Discrete Cosine Transform)}

\maketitle

\tableofcontents

\section{1D-DCT}

The \href{https://en.wikipedia.org/wiki/Discrete_cosine_transform}{DCT
  (Discrete Cosine Transform)} is an
\href{https://en.wikipedia.org/wiki/Orthonormality}{orthonormal}
\href{https://vicente-gonzalez-ruiz.github.io/transform_coding/}{transform}~\cite{vruiz__transform_coding}.

In the 1D case, the forward DCT for a digital signal $\mathbf{g}_n$ of
length $N$ is defined as~\cite{burger2016digital}
\begin{equation}
  {\mathbf h}_u = \sqrt{\frac{2}{N}}\sum_{n=0}^{N-1}{\mathbf
    g}_n{\mathbf c}_u\cos\Big(\pi\frac{u(2x+1)}{2N}\Big),
\end{equation}
for $0<u<N$, and the inverse transform is
\begin{equation}
  {\mathbf g}_n = \sqrt{\frac{2}{N}}\sum_{u=0}^{N-1}{\mathbf
    h}_u{\mathbf c}_u\cos\Big(\pi\frac{u(2x+1)}{2N}\Big),
\end{equation}
for $0<n<N$, with
\begin{equation}
  {\mathbf c}_u = \left\{
  \begin{array}{ll}
    \frac{1}{\sqrt{2}} & \quad \text{for}~u=0, \\
    1 & \quad \text{otherwise}.
  \end{array}
  \right.
\end{equation}

The transformed signal is a sequence of coefficients ${\mathbf h}_u$
with the same length than the original signal ${\mathbf g}_n$, and the
position of the coefficients in the transform domain indicate the
contribution of such coefficient to the corresponding (in this case,
spatial) frequency. For example, the coefficient at the position 0
(that is commonly refered as DC (Direct Current)) is equal to the
average of the signal. The rest of high-frequency coefficients are
called AC (Alternating Current) coefficients.

The DCT can be also expressed in matrix~\cite{sayood2017introduction}
form as
\begin{equation}
  {\mathbf h} = {\mathbf K}{\mathbf g},
\end{equation}
where ${\mathbf K}$ is the forward transform matrix. The rows of the
transform matrix are often referred to as the basis vectors for the
transform because they form an orthonormal basis set (see these
\href{https://cseweb.ucsd.edu/classes/fa17/cse166-a/lec13.pdf}{slides}),
and the elements of the transformed sequence are often called the
transform coefficients. Obviously, the inverse transform is computed
by
\begin{equation}
  {\mathbf g} = {\mathbf K}^{-1}{\mathbf h},
\end{equation}
where ${\mathbf K}^{-1}$ denotes the inverse matrix of ${\mathbf
  K}$. In the case of the DCT, ${\mathbf K}$ is
\href{https://en.wikipedia.org/wiki/Orthogonal_matrix}{orthogonal} and
therefore, ${\mathbf K}^{-1}={\mathbf K}^{\text T}$, where
$\cdot^{\text T}$ denotes the transpose of ${\mathbf K}$.


\subsection{The $\text{RGB}$-DCT}

The $\text{RGB}$-DCT is a DCT applied to a vector of $3$ elements (in
our case, a $\text{RGB}$ pixel) defined by
\begin{equation}
  \begin{bmatrix}
    \text{DCT}_0 \\
    \text{DCT}_1 \\
    \text{DCT}_2
  \end{bmatrix}
  =
  \begin{bmatrix}
    0.57735027 & 0.70710678 & 0.40824829 \\
    0.57735027 & 0.0 & -0.81649658 \\
    0.57735027 & -0.70710678 & 0.40824829
  \end{bmatrix}
  \begin{bmatrix}
    \text{R} \\
    \text{G} \\
    \text{B}
  \end{bmatrix},
\end{equation}
and the inverse transform by
\begin{equation}
  \begin{bmatrix}
    \text{R} \\
    \text{G} \\
    \text{B}
  \end{bmatrix}
  =
  \begin{bmatrix}
    0.57735027 & 0.57735027 & 0.57735027 \\
    0.70710678 & 0.0 & -0.70710678 \\
    0.40824829 & -0.81649658 & 0.40824829
  \end{bmatrix}
  \begin{bmatrix}
    \text{DCT}_0 \\
    \text{DCT}_1 \\
    \text{DCT}_2
  \end{bmatrix}.
\end{equation}
See this
\href{https://github.com/Sistemas-Multimedia/Sistemas-Multimedia.github.io/blob/master/milestones/06-YUV_compression/color-DCT_matrix.ipynb}{notebook}
to see how to compute the filter's coefficients.

When applied to the $\text{RGB}$ color domain, we will move from this
domain to the color-DCT domain that have also 3 components, that we
will denote by $\text{DCT}_0$, $\text{DCT}_1$ and $\text{DCT}_2$. If
the decorrelation is effective, most of the energy will be
concentrated in $\text{DCT}_0$, which represents the average energy of
the image (luminance).

Notice also that the DCT is orthonormal, and therefore, the matrix of
the forward transform is the transpose of the matrix of the backward
transform~\cite{sayood2017introduction}. This also means that the
contribution of the synthesis filters (which define the inverse
transform) to the reconstructed signal are independent and have
exactly the unity gain.\footnote{To find the gains of any 1D transform
we can compute the energy of the signal generated by the inverse
transform of the impulse discrete 1D signal
\begin{equation}
  \delta_{i}(x) = 
  \left\{
  \begin{array}{ll}
    1 & \text{if $i=x$}\\
    0 & \text{otherwise},
  \end{array}
  \right.
\end{equation}
where the
\href{https://en.wikipedia.org/wiki/Energy_(signal_processing)}{energy
  of a discrete signal} ${\mathbf s}$ is defined as
\begin{equation}
  \langle {\mathbf s}, {\mathbf s} \rangle =  \sum_{i}{{\mathbf s}_i^2}.
\end{equation}
}

\subsection{Quantization in the color-DCT domain}
The synthesis filters of orthonormal transforms are orthogonal (their
contributions to the reconstructed signal are independent) and have
exactly the unity gain.\footnote{The quantization error is the same in
all the subbands because all of them have exactly the same gain.}
Therefore, without considering that the entropy coding stage can
performs better for some subbands, the optimal quantization pattern
should be
\begin{equation}
  \Delta_{\text{DCT0}} = \Delta_{\text{DCT1}} = \Delta_{\text{DCT2}}.
\end{equation}
See this \href{https://github.com/Sistemas-Multimedia/Sistemas-Multimedia.github.io/blob/master/study_guide/06-color_transform/color-DCT_compression.ipynb}{notebook}.

Notice that to find the gains (of any 1D transform) we can compute the
energy of the signal generated by the inverse transform of the impulse
discrete 1D signal
\begin{equation}
  \delta_{i}(x) = 
  \left\{
  \begin{array}{ll}
    1 & \text{if $i=x$}\\
    0 & \text{otherwise},
  \end{array}
  \right.
\end{equation}
where the
\href{https://en.wikipedia.org/wiki/Energy_(signal_processing)}{energy
  of a discrete signal} $x$ is defined as
\begin{equation}
  \langle x, x\rangle =  \sum_{i}{x_i^2}.
\end{equation}

If we consider that the RD curve can be affeced by the compresiblity of the subbands, a better solution to find the optimal RD curve is:
\begin{enumerate}
\item Variying the $\Delta$, compute the RD curve for each DCT
  subband.
\item Sort the RD points (considering the three subbands at once), by
  their slope.
\item Apply progressively the combinations of quantization steps
  described by the sorted RD points.
\end{enumerate}
Notice that this algorithm is right because the DCT is orthogonal,
i.e., the contributions of the subbands to the reconstructe signal are
independent. See this \href{https://github.com/Sistemas-Multimedia/Sistemas-Multimedia.github.io/blob/master/study_guide/06-color_transform/color-DCT_compression.ipynb}{notebook}. % Arreglar link

\section{The 2D-DCT}

The 2D-DCT is separable, which means that it can be computed by
appliying the 1D-DCT to the two dimensions of the signal (a digital
image, for example). For the inverse 2D-DCT, the procedure is similar,
but appliying the inverse 1D-DWT in reverse order. The
Fig.~\ref{fig:2D-DCT_basis} shows the first 64 2D-DCT basis.

\begin{figure}
  \centering \png{2D-DCT_basis}{600} \caption{First 64 2D-DCT basis
  functions (see this
\href{https://github.com/Sistemas-Multimedia/Sistemas-Multimedia.github.io/blob/master/milestones/07-DCT/DCT_basis.ipynb}{notebook}).} % Arreglar link
  \label{fig:2D-DCT_basis}
\end{figure}

\subsection{Quantization in the (spatial) DCT domain}

The DCT is
\href{https://en.wikipedia.org/wiki/Orthonormality}{orthonormal}
(orthogonal + unitary\footnote{Unitary transforms are energy
preserving; that is, the sum of the squares of the transformed
sequence is the same as the sum of the squares of the original
sequence.}). Since the DCT is orthonormal, the gain of each subband is
equal to 1, and also that the quantization error is \emph{amplified}
with the same gain in all the subbands. However, this does not mean
that all the subbands should be quantized using the same $\Delta$,
because some subbands can be compressed more efficienly than others
and their distortion (after quantization) can be different. This
\href{https://github.com/Sistemas-Multimedia/Sistemas-Multimedia.github.io/blob/master/milestones/07-DCT/block_DCT_compression.ipynb}{notebook}
shows how to use RD optimization to determine, given a target
bit-rate, the best combination of quantization steps per subbands.

\section{Resources}

\renewcommand{\addcontentsline}[3]{}% Remove functionality of \addcontentsline
\bibliography{maths,data-compression,signal-processing,DWT,image-compression,image-processing}
