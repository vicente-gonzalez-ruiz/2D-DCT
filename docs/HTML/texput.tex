% Emacs, this is -*-latex-*-

\title{DCT (Discrete Cosine Transform)}

\maketitle

\tableofcontents

\section{1D-DCT}

The \href{https://en.wikipedia.org/wiki/Discrete_cosine_transform}{DCT
  (Discrete Cosine Transform)} is an
\href{https://en.wikipedia.org/wiki/Orthonormality}{orthonormal}
\href{https://vicente-gonzalez-ruiz.github.io/transform_coding/}{transform}~\cite{vruiz__transform_coding}.

In the 1D case, the forward DCT for a digital signal $\mathbf{g}_n$ of
length $N$ is defined as~\cite{burger2016digital}
\begin{equation}
  {\mathbf h}_u = \sqrt{\frac{2}{N}}\sum_{n=0}^{N-1}{\mathbf
    g}_n{\mathbf c}_u\cos\Big(\pi\frac{u(2x+1)}{2N}\Big),
\end{equation}
for $0<u<N$, and the inverse transform is
\begin{equation}
  {\mathbf g}_n = \sqrt{\frac{2}{N}}\sum_{u=0}^{N-1}{\mathbf
    h}_u{\mathbf c}_u\cos\Big(\pi\frac{u(2x+1)}{2N}\Big),
\end{equation}
for $0<n<N$, with
\begin{equation}
  {\mathbf c}_u = \left\{
  \begin{array}{ll}
    \frac{1}{\sqrt{2}} & \quad \text{for}~u=0, \\
    1 & \quad \text{otherwise}.
  \end{array}
  \right.
\end{equation}

The transformed signal is a sequence of coefficients ${\mathbf h}_u$
with the same length than the original signal ${\mathbf g}_n$, and the
position of the coefficients in the transform domain indicate the
contribution of such coefficient to the corresponding (in this case,
spatial) frequency. For example, the coefficient at the position 0
(that is commonly refered as DC (Direct Current)) is equal to the
average of the signal. The rest of high-frequency coefficients are
called AC (Alternating Current) coefficients.

The DCT can be also expressed in matrix~\cite{sayood2017introduction}
form as
\begin{equation}
  {\mathbf h} = {\mathbf K}{\mathbf g},
\end{equation}
where ${\mathbf K}$ is the forward transform matrix. The rows of the
transform matrix are often referred to as the basis vectors for the
transform because they form an orthonormal basis set (see these
\href{https://cseweb.ucsd.edu/classes/fa17/cse166-a/lec13.pdf}{slides}),
and the elements of the transformed sequence are often called the
transform coefficients. Obviously, the inverse transform is computed
by
\begin{equation}
  {\mathbf g} = {\mathbf K}^{-1}{\mathbf h},
\end{equation}
where ${\mathbf K}^{-1}$ denotes the inverse matrix of ${\mathbf
  K}$. In the case of the DCT, ${\mathbf K}$ is
\href{https://en.wikipedia.org/wiki/Orthogonal_matrix}{orthogonal} and
therefore, ${\mathbf K}^{-1}={\mathbf K}^{\text T}$, where
$\cdot^{\text T}$ denotes the transpose of ${\mathbf K}$.

\section{2D-DCT}

The 2D-DCT is separable, which means that it can be computed by
appliying the 1D-DCT to the two dimensions of the signal (a digital
image, for example). For the inverse 2D-DCT, the procedure is similar,
but appliying the inverse 1D-DWT in reverse order. The
Fig.~\ref{fig:2D-DCT_basis} shows the first 64 2D-DCT basis.

\begin{figure}
  \centering \png{2D-DCT_basis}{600} \caption{First 64 2D-DCT basis
  functions (see this
\href{https://github.com/Sistemas-Multimedia/Sistemas-Multimedia.github.io/blob/master/milestones/07-DCT/DCT_basis.ipynb}{notebook}).} % Arreglar link
  \label{fig:2D-DCT_basis}
\end{figure}

\subsection{Quantization in the (spatial) DCT domain}

The DCT is
\href{https://en.wikipedia.org/wiki/Orthonormality}{orthonormal}
(orthogonal + unitary\footnote{Unitary transforms are energy
preserving; that is, the sum of the squares of the transformed
sequence is the same as the sum of the squares of the original
sequence.}). Since the DCT is orthonormal, the gain of each subband is
equal to 1, and also that the quantization error is \emph{amplified}
with the same gain in all the subbands. However, this does not mean
that all the subbands should be quantized using the same $\Delta$,
because some subbands can be compressed more efficienly than others
and their distortion (after quantization) can be different. This
\href{https://github.com/Sistemas-Multimedia/Sistemas-Multimedia.github.io/blob/master/milestones/07-DCT/block_DCT_compression.ipynb}{notebook}
shows how to use RD optimization to determine, given a target
bit-rate, the best combination of quantization steps per subbands.

\section{References}

\renewcommand{\addcontentsline}[3]{}% Remove functionality of \addcontentsline
\bibliography{maths,data-compression,signal-processing,DWT,image-compression,image-processing}
